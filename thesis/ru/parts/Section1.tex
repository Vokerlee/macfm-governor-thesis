\section{Постановка задачи}
\label{sec:Section1} \index{Section1}

\subsection{Цели работы}

    Цели работы:
    \begin{enumerate}
        \item Разработать универсальную модель производительности ядра процессора учётом
        механизмов взаимодействия ядра с системой памяти.
        \item Реализовать политику регулирования тактовых частот ядра процессора в операционной
        системе Linux, используя разработанную модель производительности.
    \end{enumerate}

\subsection{Задачи}

    Задачи, которые необходимо решить для достижения целей:
    \begin{enumerate}
        \item Реализовать в операционной системе Linux:
        \begin{enumerate}
            \item Драйверы, позволяющие изменять тактовые частоты ЦП в эмулируемой системе
            архитектурным симулятором Gem5.
            \item Возможность сбора статистики микроархитектурных событий ядер ЦП, связанных с работой
            кешей, для архитектуры ARM64.
        \end{enumerate}
        \item Разработать модель производительности ядер ЦП:
        \begin{enumerate}
            \item Спроектировать теоретическую модель производительности ЦП.
            \item Реализовать набор бенчмарков, подходящий для нахождения коэффициентов
                построенной модели, и собрать для них статистику микроархитектурных событий при
                различных тактовых частотах ядер ЦП.
            \item Найти коэффициенты модели на основе измеренных данных.
        \end{enumerate}
        \item Реализовать политику регулирования тактовых частот ядер ЦП в ядре операционной системы
            Linux, используя разработанную модель.
    \end{enumerate}
\newpage
