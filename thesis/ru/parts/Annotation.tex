\begin{abstract}

    \begin{center}
        \large{Политика регулирования частот центрального процессора в ядре Linux для приложений реального времени} \\[0.5 cm]
    \large{\textit{Глаз Роман Сергеевич}} \\[1 cm]

    \end{center}

    \begin{tcolorbox}
        На данный момент для большинства устройств система памяти является узким местом,
        так как производительность ядер процессоров сильно опережает производительность системы памяти.

        Политики регулирования тактовых частот ядер процессора в операционной системе Linux,
        которая в настоящее время самым популярным ядром операционной системы,
        не учитывают, что изменение тактовой частоты не влияет на изменение времени ожидания
        инструкций процессора, обращающихся в высокие уровни кешей или ОЗУ, а времена ожидания
        всего одного обращения могут достигать сотни тактов ядра процессора.

        Отсутствие учёта устройства системы памяти для регулирования тактовых частот особенно сильно
        влияет на производительность и энергопотребление приложений реального времени, функционирование
        которых подразумевает быстрое реагирование на увеличение или уменьшение количества работы,
        выполняемой приложением.

        В данной работе предлагается политика регулирования тактовых частот, учитывающая
        устройство работы системы памяти, а также способ её реализации в ядре Linux.
    \end{tcolorbox}

\end{abstract}
\newpage
