\begin{abstract}

    \begin{center}
        \large{Политика регулирования частот центрального процессора в ядре Linux для приложений реального времени} \\[0.5 cm]
    \large{\textit{Глаз Роман Сергеевич}} \\[1 cm]

    \end{center}

    \begin{tcolorbox}
        На данный момент в мире насчитывается более 4-ёх миллиардов пользователей смартфонов.
        Несмотря на широкий диапазон возможностей смартфона, производительность,
        длительное время автономной работы и надежность являются основными критериями его
        конкурентоспособности.

        Преобладающая часть смартфонов использует архитектуру ARM big.LITTLE -- архитектура ARM/ARM64,
        основой которой является использование различных типов ядер --
        энергоэффективных с низкой производительностью и высокопроизводительных, имеющих большое
        энергопотребление.

        Также одной из основных технологий, использумых процессорами ARM, является
        Dynamic Voltage and Frequency Scaling (DVFS), благодаря которой удаётся снизить энергопотребление
        процессора, снижая его частоту работы и напряжение, если нагрузка на процессор небольшая.

        Для разработки алгоритмов планировщика операционных систем, который учитывает как наличие
        ядер разной производительности, так и технологию DVFS, очень удобным инструментом является
        симулятор микроархитектуры, позволяющий использовать подробную статистику использования
        различных компонент симулируемой системы. В данной работе используется симулятор Gem5
        и Linux в качестве исследуемой операционной системы.

        В данной работе:
        \begin{enumerate}
            \item Разработаны компоненты симулятора Gem5, предоставляющие возможность
            создавать гетерогенные (мультикластерные) архитектуры с возможностью использования ARM
            Performance Monitor Unit (PMU) счётчиков, связанных с иерархией кешей.
            \item Реализованы драйверы ядра Linux (версии 6.1) для платформ Gem5,
            предоставляющие поддержку DVFS как для процессоров (cpufreq драйвер),
            так и прочих компонент системы (компоненты devfreq драйвера).
            \item Создана модель, определяющая влияние частот процессора и прочих компонент системы
            на производительность процессора для заданной рабочей нагрузки.
            \item Разработано решение в подсистеме планировщика ядра Linux, использующее предлагаемую модель
            для контроля частоты процессорных ядер.
        \end{enumerate}
    \end{tcolorbox}

\end{abstract}
\newpage
