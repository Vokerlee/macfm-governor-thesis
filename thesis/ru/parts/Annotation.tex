\begin{abstract}

    \begin{center}
        \large{Политика регулирования частот центрального процессора в ядре Linux для приложений реального времени} \\[0.5 cm]
    \large{\textit{Глаз Роман Сергеевич}} \\[1 cm]

    \end{center}

    \begin{tcolorbox}
        На данный момент для большинства компьютерных систем память является узким местом,
        так как производительность ядер процессоров сильно опережает производительность систем памяти.

        Политики регулирования тактовых частот ядер процессора в операционной системе Linux,
        ядро которой в настоящее время является самым распространённым в мире,
        не учитывают, что изменение тактовой частоты не влияет на изменение времени ожидания
        инструкций процессора, обращающихся в высокие уровни кешей процессора или ОЗУ, а времена ожидания
        всего одного обращения могут достигать сотни тактов ядра процессора.

        Отсутствие учёта механизма взаимодействия памяти с ядром процессора для регулирования
        тактовых частот особенно сильно влияет на производительность и энергопотребление приложений
        реального времени, функционирование которых подразумевает быстрое реагирование операционной
        системы на увеличение или уменьшение нагрузки на ядро процессора.

        В данной работе предлагается политика регулирования тактовых частот, учитывающая
        механизм взаимодействия памяти с ядром процессора, а также способ её реализации в ядре
        операционной системы Linux.
    \end{tcolorbox}

\end{abstract}
\newpage
