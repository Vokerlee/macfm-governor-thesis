\section{Заключение}
\label{sec:Section5} \index{Section5}

    В данной работе были выполнены все поставленные цели и решены следующие задачи:
    \begin{enumerate}
        \item Проведено исследование существующих политик регулирования тактовых частот ядер
        центрального процессора, выявлены узкие места таких политик, проведён обзор литературы на тему
        альтернативных способов регулирования тактовых частот.
        \item Разработана модель производительности ядер центрального процессора, учитывающая архитектуру
        взаимодействия ядер с системой памяти.
        \item Разработан способ применения модели производительности к политике регулирования тактовых частот.
        \item Модифицирован симулятор Gem5 для возможности измерения статистики микроархитектурных событий,
        связанных с кешами процессора, из пространства пользователя и из пространства ядра в операционной
        системе Linux.
        \item Разработан набор бенчмарков, позволяющих вычислить коэффициенты в модели производительности.
        \item Собраны статистические данные счётчиков микроархитектурных событий, необходимых для обеспечения
        работоспособности модели производительности, для набора бенчмарков c помощью симулируемой системы с
        процессором архитектуры ARM64.
        \item Вычислены коэффициенты в модели производительности на основе собранных статистических данных.
        \item В ядре операционной системы Linux версии 6.1:
        \begin{enumerate}
            \item Реализованы интерфейсы драйверов $clk$, $cpufreq$, $devfreq$ для обеспечения возможности
            изменения тактовых частот ядер процессора и прочих компонент симулируемой системы Gem5
            как из пространства ядра, так и из пространства пользователя.
            \item Реализована политика регулирования тактовых частот ядер центрального процессора.
        \end{enumerate}
    \end{enumerate}

    Планируемые исследования в будущем по данной теме:
    \begin{enumerate}
        \item Применение современных методов машинного обучения для создания более качественного
        алгоритма регулирования коэффициента параллелизма обращений в память в модели производительности ядер.
        \item Построение модели энергопотребления процессора и оперативной памяти.
        \item Разработка отдельной политики регулирования тактовых частот оперативной памяти на основе
        построенной модели производительности.
        \item Разработка политики рекомендаций планировщику ядра Linux о размещении потоков исполнения на ядрах
        процессора на основе информации коэффициентов в модели производительности.
    \end{enumerate}

\newpage
