\section*{Приложение}
\addcontentsline{toc}{section}{Приложение}
\label{sec:Apendix} \index{Apendix}

\subsection*{Список используемых обозначений и терминов}

\begin{enumerate}
    \item CPU (Central Processing Unit) -- ЦП (центральный процессор);
    \item CPU core -- ядро центрального процессора;
    \item $cycles$ -- количество циклов процессорного ядра за определённый промежуток времени;
    \item $instrs$ -- количество исполненных инструкций процессорным ядром за определённый
        промежуток времени;
    \item $ipc = \frac{instrs}{cycles}$, $cpi = \frac{cycles}{instrs}$;
    \item Pipeline -- конвейер (обычно применятся в контексте схемы исполнения инструкции
        процессорным ядром);
    \item OoO (Out-of-Order) -- парадигма в современных высокопроизводительных CPU, основанная
        на спекулятивном исполнении инструкций для повышения значения $ipc$;
    \item ROB (Re-order buffer) -- циклический буфер, используемый для работы алгоритма Томасуло
        для поддержки OoO в CPU;
    \item Data hazard -- зависимость по данным (употребляется в контексте использования
        одинаковых регистров в соседних инструкциях, из-за чего инструкции необходимо исполнить
        строго в порядке их логического следования).
    \item SoC (System-on-Chip) -- интегральная схема, включающая в себя большинство или все
        компоненты компьютерной системы, такие как CPU, интерфейсы памяти, устройства ввода-вывода
        и так далее;
    \item GPU (Graphics Processing Unit) -- графический процессор;

    \item Latency -- задержка (латентность) в рамкам какого-либо события (например,
        обращения в память), выраженная во времени или циклах какого-то устройства;
    \item Throughput -- утилизация устройства, выраженная в единицах транзакций, связанных с
        этим устройством, в единицу времени;
    \item Bandwidth -- пропускная способность устройства (максимально возможная утилизация
        устройства);
    \item Workload -- определённая рабочая нагрузка, исполняемая на устройстве;
    \item DDR SDRAM (Double Data Rate Synchronous Dynamic Random-Access Memory) -- класс
        интегральных схем памяти, используемой в компьютерах (обычно в качестве оперативной
        памяти);
    \item LPDDR SDRAM (Low-Power Double Data Rate SDRAM) -- класс интегральных
        схем памяти, спецификой которого является низкое потребление энергии (как правило,
        используемый в мобильных компьютерах);
    \item DDR (Double Data Rate) -- общее обозначение, которое подразумевает DDR SDRAM или
        LPDDR SDRAM память;

\end{enumerate}
