\section*{Приложение}
\addcontentsline{toc}{section}{Приложение}
\label{sec:Apendix} \index{Apendix}

\subsection*{Список используемых обозначений и терминов}

\begin{enumerate}
    \item $cycles$ -- количество тактов процессорного ядра за определённый промежуток времени;
    \item $instrs$ -- количество исполненных инструкций процессорным ядром за определённый
        промежуток времени;
    \item $ipc = \frac{instrs}{cycles}$, $cpi = \frac{cycles}{instrs}$;
    \item OoO (Out-of-Order) -- парадигма в современных высокопроизводительных CPU, основанная
        на спекулятивном исполнении инструкций для повышения значения $ipc$;
    \item ROB (Re-order buffer) -- циклический буфер, используемый для работы алгоритма Томасуло
        для поддержки OoO в CPU;
    \item SoC (System-on-Chip) -- интегральная схема, включающая в себя большинство или все
        компоненты компьютерной системы, такие как CPU, интерфейсы памяти, устройства ввода-вывода
        и так далее;
    \item DDR SDRAM (Double Data Rate Synchronous Dynamic Random-Access Memory) --
        динамическая оперативная память синхронного доступа с двойной скоростью передачи данных --
        класс интегральных схем памяти, используемой в
        компьютерах (в качестве ОЗУ);
    \item LPDDR SDRAM (Low-Power Double Data Rate Synchronous Dynamic Random-Access Memory) --
        динамическая оперативная память синхронного доступа с двойной скоростью передачи данных и
        с низким энергопотреблением -- класс интегральных схем памяти, используемой в
        мобильных компьютерах (в качестве ОЗУ);
    \item DDR (Double Data Rate) -- общее обозначение, которое подразумевает DDR SDRAM или
        LPDDR SDRAM память;
    \item DVFS (Dynamic Voltage-Frequency Scaling) -- технология, позволяющая регулировать
        рабочие частоту и напряжение устройства в режиме реального времени.

\end{enumerate}
