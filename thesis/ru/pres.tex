\documentclass{beamer}
\usepackage[utf8]{inputenc}
\usepackage{comment}
% \usetheme{Madrid}
\usepackage[maxbibnames=99]{biblatex}
\definecolor{cvut_navy}{HTML}{0065BD}
\definecolor{cvut_blue}{HTML}{6AADE4}
\definecolor{cvut_gray}{HTML}{156570}
\definecolor{faeng_blue}{HTML}{09246F}
\usepackage[makeroom]{cancel}
\usepackage{threeparttable}

\setbeamercolor{section in toc}{fg=black,bg=yellow}
\setbeamercolor{alerted text}{fg=cvut_blue}
\usepackage{tikzsymbols}
\usepackage{textcomp}
\usepackage{parskip}
\definecolor{darkblue}{rgb}{0, 0, 0.5}
\definecolor{babyblue}{rgb}{0.54, 0.81, 0.94}
\usepackage{pgf}
\usepackage{color,soul}
\usepackage{tcolorbox}
\tcbuselibrary{skins}
\usepackage{minted}
\usepackage{hyperref}
\usepackage{xcolor,soul}
\definecolor{lightblue}{rgb}{.90,.95,1}
\sethlcolor{lightblue}

\renewcommand<>{\hl}[1]{\only#2{\beameroriginal{\hl}}{#1}}
\setbeamertemplate{page number in head/foot}[totalframenumber]

\usepackage{empheq}
\usepackage{xcolor}
\definecolor{lightgreen}{HTML}{90EE90}
\newcommand{\boxedeq}[2]{\begin{empheq}[box={\fboxsep=6pt\fbox}]{align}\label{#1}#2\end{empheq}}
\newcommand{\coloredeq}[2]{\begin{empheq}[box=\colorbox{lightgreen}]{align}\label{#1}#2\end{empheq}}
\newcommand{\highlight}[1]{%
  \colorbox{red!40}{$\displaystyle#1$}}

\definecolor{babyblue}{rgb}{0.54, 0.81, 0.94}
\definecolor{babypink}{rgb}{0.96, 0.76, 0.76}
\definecolor{blue(ncs)}{rgb}{0.0, 0.53, 0.74}
\definecolor{pistachio}{rgb}{0.58, 0.77, 0.45}
\definecolor{darksalmon}{rgb}{0.91, 0.59, 0.48}
\definecolor{lightsalmonpink}{rgb}{1.0, 0.6, 0.6}
\definecolor{columbiablue}{rgb}{0.61, 0.87, 1.0}
\definecolor{corn}{rgb}{0.98, 0.93, 0.36}
\definecolor{jonquil}{rgb}{0.98, 0.85, 0.37}
\definecolor{bananayellow}{rgb}{1.0, 0.88, 0.21}
\newcommand{\bert}{\ensuremath{%
  \mathchoice{\includegraphics[height=2ex]{Bert-pic-removebg-preview.png}}
    {\includegraphics[height=2ex]{Bert-pic-removebg-preview.png}}
    {\includegraphics[height=1.5ex]{Bert-pic-removebg-preview.png}}
    {\includegraphics[height=1ex]{Bert-pic-removebg-preview.png}}
}}

\useoutertheme{infolines}

\usepackage{courier}
%\usepackage{animate}
\usepackage{expl3}
%\usepackage[listings,theorems]{tcolorbox}

\newcommand\scroll[4][§§]{
  \seq_set_split:Nnn\g_inputline_seq{#1}{#4}
  \seq_map_inline:Nn\g_inputline_seq{
    \seq_gput_right:Nx\g_linebuffer_seq{##1}
    \int_compare:nT{\seq_count:N\g_linebuffer_seq>#3}{
      \seq_gpop_left:NN\g_linebuffer_seq\dummy
    }
  }
  \fbox{\begin{minipage}[t][#3\baselineskip]{#2}
    \ttfamily
    \seq_map_inline:Nn\g_linebuffer_seq{\mbox{##1}\\}
  \end{minipage}}
}
\newcommand\clearbuf{\seq_gclear:N\g_linebuffer_seq}
\ExplSyntaxOff

\setbeamertemplate{headline}{%
\begin{beamercolorbox}[colsep=1.5pt]{upper separation line head}
\end{beamercolorbox}
\begin{beamercolorbox}{section in head/foot}
    \vskip2pt\insertsectionnavigationhorizontal{\paperwidth}{}{\hskip0pt plus1filll}\vskip2pt
\end{beamercolorbox}%

\begin{beamercolorbox}[colsep=1.5pt]{lower separation line head}
\end{beamercolorbox}
}
\makeatletter
\newcommand\SoulColor{%
  \let\set@color\beamerorig@set@color
  \let\reset@color\beamerorig@reset@color}
\makeatother
\SoulColor
\usepackage{amsmath, bm}
\usepackage{tikz}
\setbeamercovered{dynamic}

\newcommand{\highlightt}[1]{%
  \colorbox{blue!40}{$\displaystyle#1$}}

\newenvironment<>{problock}[1]{%
  \begin{actionenv}#2%
      \def\insertblocktitle{#1}%
      \par%
      \mode<presentation>{%
       % \setbeamercolor{block title}{fg=white,bg=orange!20!black}
        %\setbeamercolor{block title}{fg=white,bg=red!10!black}
        \setbeamercolor{block title}{fg=white,bg=babyblue}
       \setbeamercolor{block body}{fg=black,bg=white!50}
       \setbeamercolor{itemize item}{fg=orange!20!black}
       \setbeamertemplate{itemize item}[triangle]
     }%
      \usebeamertemplate{block begin}
    \par\usebeamertemplate{block end}
    \end{actionenv}
    }

    \newcommand<>{\uncovergraphics}[2][{}]{
    % Taken from: <https://tex.stackexchange.com/a/354033/95423>
    \begin{tikzpicture}
    \node[anchor=south west,inner sep=0] (B) at (4,0)
        {\includegraphics[#1]{#2}};
    \alt#3{}{%
        \fill [draw=none, fill=background, fill opacity=0.9] (B.north west) -- (B.north east) -- (B.south east) -- (B.south west) -- (B.north west) -- cycle;
    }
    \end{tikzpicture}
}

\newlength\dlf
\newcommand\alignedbox[3][yellow]{
  % #1 = color (optional, defaults to yellow)
  % #2 = before alignment
  % #3 = after alignment
  &
  \begingroup
  \settowidth\dlf{$\displaystyle #2$}
  \addtolength\dlf{\fboxsep+\fboxrule}
  \hspace{-\dlf}
  \fcolorbox{red}{#1}{$\displaystyle #2 #3$}
  \endgroup
}

\usepackage{collcell}
%\usepackage{remreset}% tiny package containing just the \@removefromreset command
\makeatletter

\usepackage{xcoffins}
\NewCoffin\tablecoffin
\NewDocumentCommand\Vcentre{m}
  {%
    \SetHorizontalCoffin\tablecoffin{#1}%
    \TypesetCoffin\tablecoffin[l,vc]%
  }

\usepackage{pgfpages}

%% Important
%%% show note or disable note.
%\setbeameroption{show notes on second screen=right} % Both

\setbeamertemplate{note page}{\pagecolor{gray!5}\insertnote}\usepackage{palatino}
\usepackage{xcolor}
\usepackage{soul}
\usepackage{etoolbox}
\makeatletter
%\patchcmd{\slideentry}{\ifnum#2>0}{\ifnum2>0}{}{\@error{unable to patch}}% replace the subsection number test with a test that always returns true
\makeatother

%\setbeamercolor*{palette primary}{bg=cvut_navy,fg=gray!20!white}
\setbeamercolor*{palette primary}{bg=faeng_blue,fg=gray!20!white}
\setbeamercolor*{palette secondary}{bg=faeng_blue,fg=gray!20!white} % no color
%\setbeamercolor*{palette secondary}{bg=cvut_navy,fg=cvut_navy}
%\setbeamercolor*{palette secondary}{bg=cvut_blue,fg=white}
\setbeamercolor*{palette tertiary}{parent=palette primary} % color of the top and date
\setbeamercolor*{palette quaternary}{fg=faeng_blue,bg=gray!5!white}
\setbeamercolor*{sidebar}{fg=faeng_blue,bg=gray!15!white}
\usepackage[first=0,last=9]{lcg}
\usepackage{color, colortbl}
\usepackage{stackengine,tikz}
\usepackage{transparent}
\colorlet{Gray}{gray!30}
\newcommand*{\MinNumber}{0}%
\newcommand*{\MaxNumber}{0.4}%
\definecolor{bubblegum}{rgb}{0.99, 0.76, 0.8}
\newcommand{\ApplyGradient}[1]{%
  \pgfmathsetmacro{\PercentColor}{100.0*(#1-\MinNumber)/(\MaxNumber-\MinNumber)}%
  \edef\x{\noexpand\cellcolor{babyblue!\PercentColor}}\x\textcolor{black}{#1}%
}
\newcolumntype{R}{>{\collectcell\ApplyGradient}{c}<{\endcollectcell}}

\setbeamercolor{titlelike}{parent=palette primary}
\setbeamercolor{frametitle}{parent=palette primary}

\setbeamercolor{B}{bg=red!30,fg=black}

\setbeamertemplate{section in toc}[default]
\setbeamercolor{itemize item }{fg=blue}
\setbeamertemplate{itemize item}[circle]

\setbeamercolor*{separation line}{}
\setbeamercolor*{fine separation line}{}

\setbeamertemplate{navigation symbols}{}
\setbeamertemplate{caption}{\raggedright\insertcaption\par}

\setbeamercolor*{block title example}{fg=white,bg=purple!75!black}
\setbeamercolor*{block body example}{fg= black, bg= white}

\setbeamercolor{itemize item}{fg=cvut_navy} % all frames will have red bullets
\setbeamercolor{block title}{bg=red!30,fg=black}
\setbeamertemplate{subsection in toc}[subsections numbered]

\usepackage{eqnarray,amsmath}
\usepackage{amsfonts}
\usepackage{amssymb}
\usefonttheme{professionalfonts}
\usepackage{graphicx}
\usepackage{booktabs}
\usepackage{bm}
\usepackage{mathtools}
\usepackage[utf8]{inputenc}
\usepackage[T1]{fontenc}
\usepackage[russian]{babel}
\usepackage{lmodern}

\setbeamercolor{block title}{fg=black, bg=yellow}
\setbeamercolor{block2}{use=structure,fg=white,bg=purple!75!black}

\def\bq{\mbox{\kern.1ex\protect\raisebox{-1.3ex}[0pt][0pt]{''}\kern-.1ex}}
\def\eq{\mbox{\kern-.1ex``\kern.1ex}}
\def\ifundefined#1{\expandafter\ifx\csname#1\endcsname\relax }%
\ifundefined{uv}%
        \gdef\uv#1{\bq #1\eq}
\fi

\usepackage{algpseudocode}
\usepackage{ragged2e}
\usepackage{csquotes}

\usepackage{biblatex}
\addbibresource{referencias.bib}


%====================================================
%========== DEFINITION OF AUTHORS ETC...=============
%====================================================

\author[Глаз Роман Сергеевич]{Глаз Роман Сергеевич}
\institute[]{
    Московский физико-технический институт  \\
    Кафедра микропроцессорных технологий в интеллектуальных системах управления
    \vspace{1mm} \\
    \textbf{Научный руководитель:}  \\
    Кринов Пётр Сергеевич, к. ф.-м. н.
}

\title[]{Политика регулирования частот центрального процессора в ядре Linux для приложений реального времени}

\date[\today]{\small{Выпускная квалификационная работа бакалавра} \\ \small{\today}}

%====================================================
%========== BEGINNING OF DOCUMENT ===================
%====================================================

\begin{document}

\begin{frame}[plain]
	\titlepage
	\begin{center}%
  		\includegraphics[height=1.5cm]{images/mipt_logo.png}
	\end{center}
\end{frame}

\begin{frame}[plain]{Содержание}
\tableofcontents[]
\note[item]{note here }
\end{frame}

% =========================================

\section{Введение}

\begin{frame}{Введение}
    Este \textit{tamplate} é feito com a logo oficial da FAENG-UFMT. É uma versão modificada do \textit{tamplate}
    da Universidade Politécnica da Catalunha (UPC). \href{https://www.overleaf.com/latex/templates/upc-beamer-template/wywdgpnbstkw}{\beamergotobutton{Link}}

    \vspace{1cm}

    Para aprender a construir apresentações em LaTeX, veja o seguinte tutorial:
    \href{https://www.overleaf.com/learn/latex/Beamer_Presentations\%3A_A_Tutorial_for_Beginners_(Part_1)\%E2\%80\%94Getting_Started}{\beamergotobutton{Link}}
\end{frame}

% =========================================

% \section{FFF}

% \begin{frame}{Figura 2}
%     \begin{figure}
%         \centering
%         \includegraphics[width=6.5cm]{Beamer/figs/teste.png}
%         \caption{Figura gerada por Inteligência Artificial}
%         \label{fig:AIimage}
%     \end{figure}
% \end{frame}

% =========================================

\begin{frame}{Tabelas}
    \begin{table}[]
        \centering
        \begin{tabular}{|c|c|c|}
            \hline
            Modelo & Acurácia \\
            \hline
            SVM & 0.88 \\
            \textit{k}-NN & 0.80 \\
            Naive Bayes & 0.83 \\
            MLP & 0.95 \\
            \hline
        \end{tabular}
        \caption{Exemplo de Tabela}
        \label{tab:my_label}
    \end{table}
\end{frame}

% =========================================

% \begin{frame}{Listagem}
%     Listagem de itens com \textit{itemize}:
%     \begin{itemize}
%         \item Item 1
%         \item Item 2
%         \item Item 3
%     \end{itemize}
%     \vspace{1cm}
%     Listagem de itens com \textit{enumerate}:
%     \begin{enumerate}
%         \item Item 1
%         \item Item 2
%         \item Item 3
%     \end{enumerate}
% \end{frame}

% =========================================

% \begin{frame}{Blocos}
%     Lorem ipsum dolor sit amet, consectetur adipiscing elit. Etiam eget ligula eu lectus lobortis condimentum. Aliquam nonummy auctor massa.

%     \begin{alertblock}{Bloco}
%         Pellentesque habitant morbi tristique senectus et netus et malesuada fames ac turpis egestas. Nulla at risus.
%     \end{alertblock}

%     Quisque purus magna, auctor et, sagittis ac, posuere eu, lectus. Nam mattis, felis ut adipiscing.
% \end{frame}

% =========================================

% \begin{frame}{Notação Matemática}
%     \begin{alertblock}{Distribuição Normal}
%         \begin{equation*}
%         f(x) = \frac{1}{\sigma\sqrt{2 \pi}} e^{-\frac{1}{2}\left(\frac{x-\mu}{\sigma}\right)^2}
%         \end{equation*}
%     \end{alertblock}

%     \begin{alertblock}{Transformada de Fourier Discreta}
%         \begin{equation*}
%         X(\omega) = \sum^{L-1}_{n=0} x_n e^{-i\omega n}, \quad x \in \mathbb{R}^L
%         \end{equation*}
%     \end{alertblock}
% \end{frame}

% =========================================

% \begin{frame}{Algoritmos}
%     Algoritmo \textit{Selection Sort}, que ordena um vetor $x\in\mathbb{R}^N$ em tempo $\mathcal{O}(N^2)$:

%     \vspace{0.5cm}

%     \begin{algorithmic}
%         \For{$i \in [0,...,N-1]$}
%             \State $m \gets i$
%             \For{$j \in [i+1,...,N-1]$}
%                 \If{$x_j < x_m$}
%                     \State $m \gets j$
%                 \EndIf
%             \EndFor
%             \If{$x_i \neq x_m$}
%                 \State $a \gets x_i$
%                 \State $x_i \gets x_m$
%                 \State $x_m \gets a$
%             \EndIf
%         \EndFor
%     \end{algorithmic}
% \end{frame}

% =========================================

% \begin{frame}[fragile]{Código-fonte}
%     Código-fonte do algoritmo \textit{Selection Sort} em Python:

%     \begin{verbatim}
% lista = [3,2,1]
% for i in range(len(lista)):
%     menor = i
%     for j in range(i+1,len(lista)):
%         if lista[j] < lista[menor]:
%                 menor = j
%     if lista[i] != lista[menor]:
%             aux = lista[i]
%             lista[i] = lista[menor]
%             lista[menor] = aux
% print(lista)
%     \end{verbatim}

%     \alert{Utilize ``fragile'' ao invés de ``plain'' ao definir um \textit{frame} que contém um \textit{verbatim}}
% \end{frame}

% \section{Citações}

% \begin{frame}{Citação indireta}

%     \textbf{Exemplo 1}

%     \justify
%     Quando existem alegações que contradizem conhecimentos bem consolidados, devem haver evidências criteriosamente inspecionadas e fortes o suficiente para que sejam abandonadas as evidências que já estavam estabelecidas \cite{sagan-1980}.

%     \textbf{Exemplo 2}

%     Segundo \citeauthor{sagan-1980} (\citeyear{sagan-1980}) \cite{sagan-1980}, quando existem alegações que contradizem conhecimentos bem consolidados, devem haver evidências criteriosamente inspecionadas e fortes o suficiente para que sejam abandonadas as evidências que já estavam estabelecidas.

% \end{frame}

% \begin{frame}{Citação direta}

%     \textbf{Citação direta curta}

%     Segundo \citeauthor{sagan-1980} (\citeyear{sagan-1980}) \cite{sagan-1980}, ''Alegações extraordinárias requerem evidências extraordinárias''.
%     \\

%     \textbf{Citação direta longa}\\

%     \begin{quote}
%         ''Quando expostas à luz ultravioleta as moléculas orgânicas que constituem oda a vida sobre a Terra se desfazem ou formam ligações químicas nocivas. Entre os seres que habitam os oceanos, os mais difundidos são minúsculas plantas unicelulares que flutuam perto da superfície da água — os fitoplanctons.'' \cite{sagan1998bilhoes}
%     \end{quote}

% \end{frame}

% =========================================

\section{Заключение}

% =========================================

\begin{frame}
    \begin{center}
    \Huge \textcolor{faeng_blue}{Obrigado!}
    \end{center}
\end{frame}

% =========================================

\section{Библиография}

\begin{frame}{Библиография}
    %\bibliography{Beamer/referencias}
    \printbibliography
\end{frame}


\end{document}
% =============================================================
% =========================== END =============================
% =============================================================
