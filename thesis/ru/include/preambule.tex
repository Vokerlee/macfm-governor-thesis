%%% For Russian language
\usepackage{cmap}
\usepackage{mathtext}
\usepackage[T2A]{fontenc}
\usepackage[utf8]{inputenc}
\usepackage[russian]{babel}

%%% For maths
\usepackage{amsmath,amsfonts,amssymb,amsthm,mathtools}
\usepackage{icomma}

%% Formulas' enumaeration
%\mathtoolsset{showonlyrefs=true} % only if \eqref{} uses the formula.
%\usepackage{leqno}

%% Fonts
\usepackage{euscript}
\usepackage{mathrsfs}

%% Page geometry
\usepackage[left=3cm,right=1.5cm,top=2cm,bottom=2cm,bindingoffset=0cm]{geometry}

%% Russian lists
\usepackage{enumitem}
\makeatletter
\AddEnumerateCounter{\asbuk}{\russian@alph}{щ}
\makeatother

\usepackage{tcolorbox}
\usepackage{url}
\usepackage{svg}

\usepackage[ddmmyyyy,hhmmss]{datetime}
\renewcommand{\dateseparator}{.}

%%% For images
\usepackage{caption}
\captionsetup{justification=centering} % центрирование подписей к картинкам
\usepackage{graphicx}                  % Для вставки рисунков
\graphicspath{{images/}{images2/}}     % папки с картинками
\setlength\fboxsep{3pt}                % Отступ рамки \fbox{} от рисунка
\setlength\fboxrule{1pt}               % Толщина линий рамки \fbox{}
\usepackage{wrapfig}                   % Обтекание рисунков и таблиц текстом

%%% Работа с таблицами
\usepackage{array,tabularx,tabulary,booktabs} % Useful features for work with tables
\usepackage{longtable}                        % Long tables
\usepackage{multirow}                         % Merge string in tables

%% Red string
\setlength{\parindent}{2em}
\usepackage{indentfirst}

%% Ranges
\linespread{1}
\usepackage{multirow}

%% TikZ
\usepackage{tikz}
\usetikzlibrary{graphs,graphs.standard}

%% Верхний колонтитул
\usepackage{fancyhdr}
\pagestyle{fancy}

%% Перенос знаков в формулах (по Львовскому)
\newcommand*{\hm}[1]{#1\nobreak\discretionary{}{\hbox{$\mathsurround=0pt #1$}}{}}

%% дополнения
\usepackage{float}   % Добавляет возможность работы с командой [H] которая улучшает расположение на странице
\usepackage{gensymb} % Красивые градусы
\usepackage{caption} % Пакет для подписей к рисункам, в частности, для работы caption*
\usepackage{listings} % Пакет для листингов с кодом
\usepackage{xcolor}
\usepackage[noadjust]{cite}

\definecolor{codegreen}{rgb}{0,0.6,0}
\definecolor{codegray}{rgb}{0.5,0.5,0.5}
\definecolor{codepurple}{rgb}{0.58,0,0.82}
\definecolor{backcolour}{rgb}{0.95,0.95,0.92}

\lstdefinestyle{mystyle}{
    backgroundcolor=\color{backcolour},
    commentstyle=\color{codegreen},
    keywordstyle=\color{magenta},
    numberstyle=\tiny\color{codegray},
    stringstyle=\color{codepurple},
    basicstyle=\ttfamily\small,
    breakatwhitespace=true,
    columns=flexible,
    breaklines=true,
    captionpos=t,
    keepspaces=true,
    numbers=left,
    numbersep=5pt,
    showspaces=false,
    showstringspaces=false,
    showtabs=false,
    tabsize=4
}

\lstset{style=mystyle}

% Hyperrefs inside pdf
\usepackage[unicode, pdftex]{hyperref}
